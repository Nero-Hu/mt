\documentclass[11pt]{article}

\usepackage{graphicx}
\usepackage{wrapfig}
\usepackage{url}
\usepackage{wrapfig}
\usepackage{color}
\usepackage{marvosym}
\usepackage{enumerate}
\usepackage{subfigure}
\usepackage{tikz}
\usepackage{amsmath}
\usepackage{amssymb}
\usepackage{hyperref}
\usepackage{CJKutf8}
\usepackage{listings}
\usepackage[encapsulated]{CJK}
% set the default code style
\lstset{
    frame=tb, % draw a frame at the top and bottom of the code block
    tabsize=2
}

\oddsidemargin 0mm
\evensidemargin 5mm
\topmargin -20mm
\textheight 240mm
\textwidth 160mm

\newcommand{\vw}{{\bf w}}
\newcommand{\vx}{{\bf x}}
\newcommand{\vy}{{\bf y}}
\newcommand{\vxi}{{\bf x}_i}
\newcommand{\yi}{y_i}
\newcommand{\vxj}{{\bf x}_j}
\newcommand{\vxn}{{\bf x}_n}
\newcommand{\yj}{y_j}
\newcommand{\ai}{\alpha_i}
\newcommand{\aj}{\alpha_j}
\newcommand{\X}{{\bf X}}
\newcommand{\Y}{{\bf Y}}
\newcommand{\vz}{{\bf z}}
\newcommand{\msigma}{{\bf \Sigma}}
\newcommand{\vmu}{{\bf \mu}}
\newcommand{\vmuk}{{\bf \mu}_k}
\newcommand{\msigmak}{{\bf \Sigma}_k}
\newcommand{\vmuj}{{\bf \mu}_j}
\newcommand{\msigmaj}{{\bf \Sigma}_j}
\newcommand{\pij}{\pi_j}
\newcommand{\pik}{\pi_k}
\newcommand{\D}{\mathcal{D}}
\newcommand{\el}{\mathcal{L}}
\newcommand{\N}{\mathcal{N}}
\newcommand{\vxij}{{\bf x}_{ij}}
\newcommand{\vt}{{\bf t}}
\newcommand{\yh}{\hat{y}}
\newcommand{\code}[1]{{\footnotesize \tt #1}}
\newcommand{\alphai}{\alpha_i}

\pagestyle{myheadings}
\markboth{Haitang Hu}{Machine Translation : Homework 4}


\title{Machine Translation\\Rerank}
\author{Haitang Hu \\
  {\tt hthu@cs.jhu.edu}}
\date{\today}

\begin{document}
\large
\maketitle
\thispagestyle{headings}
\section{Rerank Methods} % (fold)
\label{sec:rerank_methods}
\subsection{Minimum Bayes-Risk Decoding} % (fold)
\label{sub:minimum_bayes_risk_decoding}
Minimum Bayes-Risk Decoding\cite{mbr} aims to minimize the expected loss of translation errors. By applying specific metrics(\textbf{BLEU}, \textbf{WER}) as loss function, we could form our MBR decoder to improve reranking performance on Machine Translation System.\\
The minimum Bayes-Risk can be expressed in terms of \textit{loss function} and our decoder generated results $\delta(F)$.
$$ R(\delta(F)) = E_{P(E,A,F)}[L((E,A),\delta(F))]$$
Where the expectation is taken under the distribution of $P(E,A,F)$, which means the true joint distribution.\\
For this problem, we have a well known solution if given the loss function and a distribution
$$ \delta(F) = \text{arg min}_{E^{\prime}, A^{\prime}} \sum_{E,A}L((E,A),(E^{\prime}, A^{\prime}))P(E,A|F) $$
where $E$ is the translation sentence, $A$ is a alignment under the translation $(E,F)$. But this ideal model is far from reality, since we don't have the true distribution for our $P(E,A|F)$. Here, we compromise to our statiscal methods, to guess the distribution through $N-best$ list we have, and now the model becomes
$$ \hat{i} = \text{arg min}_{i \in \{1,2,\dots, N\}} \sum_{j=1}^N L((E_j,A_j), (E_i,A_i))P(E_j,A_j|F)$$
where $P(E_j,A_j|F)$ can be represented as 
$$ P(E_j,A_j|F) = \frac{P(E_j,A_j,F)}{\sum_j^NP(E_j,A_j,F)}$$
Note that $P(E_j,A_j,F)$ now is just a empirical distribution under given $N-best$ list.\\
This model suggests that we should look into all our $N-best$ list, and select the \textit{average} one as our results, since the \textit{average} one always gives us less surprise, or \textit{risk}.\\
Also, it might be worth citing the paper's proof, that if we use a indicator function on our loss function, then MBR reduced to the MAP estimator.
$$ \delta_{MAP}(F) = \text{arg max}_{(E^{\prime}, A^{\prime})}P(E^{\prime}, A^{\prime}|F)$$
This is intuitive, since MAP just use point estimate which assumes all our distribution density peaks at the point. Instead, MBR gives a more smoothed distribution.
% subsection minimum_bayes_risk_decoding (end)

\subsection{Feature Extension} % (fold)
\label{sub:feature_extension}
It is natural that we should not only depends on our translation model, language model and lexical model score. Here we encode another 2 belief into our features, to get a better representation of our domain knowledge. First, we consider the word counts, an intuitive way to encode our belief is to penalize with respect to the difference of length $\delta(c, r)$ between candidate and reference. The second feature is simply the number of untranslated Russian words, notated as $u(c)$. So, we have our model score to be following
$$ s(c) = \lambda_{l(c)}l(c) + \lambda_{t(c)}t(c) + \lambda_{lex(c)}lex(c) - \lambda_{\delta(c, r)}\delta(c, r) - \lambda_{u(c)}u(c)$$
Here we have 5 parameters, and we should choose them to fit best to our training data.
% subsection feature_extension (end)
% section rerank_methods (end)

\section{Implementation} % (fold)
\label{sec:implementation}
\subsection{Metric} % (fold)
\label{sub:metric}
Generally, \textbf{BLEU} will be used as our loss function. Since \textbf{BLEU} score always lies in the range of $(0,1)$, so we could encode our loss function to be
$$ L((E_j,A_j), (E_i,A_i)) = 1 - BLEU((E_j,A_j), (E_i,A_i))$$
Recall, we also need to specify our posterior distribution. Here we specify it to be
$$ P(E_j,A_j|F) = \log(l(E_j)) +  \log(t(E_j|F)) + \log(lex(E_j,A_j|F))$$
Also, there is another point worth mentioning, that is the \textbf{BLEU} can both applied on \textit{string level} and \textit{word level}. We will show the performance comparison later.
% subsection metric (end)
\subsection{Efficiency} % (fold)
\label{sub:efficiency}
Since for each $N-best$ list, we need to at least loop $N^2$ times, since we have a pairwise loss, so it is neccesary to implement it in a smarter way. Here we employ a matrix method to avoid to loop twice for computing normalize constant and pariwise loss.
% subsection efficiency (end)
% section implementation (end)

\section{Evaluation} % (fold)
\label{sec:evaluation}
\subsection{Result} % (fold)
\label{sub:result}
\begin{table}[!htf]
\centering
\begin{tabular}{ | c | c |}
\hline
Method & Score\\
\hline
baseline & $0.2735$\\
\hline
baseline($lm = -1.0, tm = -0.65, lex = -1.3$) & $0.2817$ \\
\hline
feature ext($lm = -1.0, tm = -0.65, lex = -1.3, c = 1.03, u = 0.1$) & $0.2893$ \\
\hline
MBR & $0.2916$ \\
\hline
MBR + word count & $\bf{0.2918}$ \\
\hline
\end{tabular}
\caption{Result}
$c, u$ stands for word count weight and untranslated words weight 
\end{table}
% subsection result (end)
% section evaluation (end)

\subsection{Evaluation and Optimization} % (fold)
\label{sub:evaluation_Optimization}
As we can see here, simply tuning the parameter of \textit{baseline} system gives us a big improvement, which shows the huge gain we could get from \textit{MERT} or \textit{PRO}(But I did not encode them).\\
Next, we look at he feature extension, which again raise a lot score, since we actually encode the significant reason to express our belief, which happens to be right.\\
We gain the best score by using MBR method with counting \textit{word count} feature into our posterior distribution, this shows the we could benifit from more feature under MBR setting. Also, since 
% subsection evaluation (end)

\begin{thebibliography}{50}
\bibitem{mbr} Shankar Kumar and William Byrne. \textsl{Minimum Bayes-Risk Decoding for Statistical Machine Translation}, 2011.

\end{thebibliography}
\end{document}
